% start of file `template.tex'.
% This work may be distributed and/or modified under the
% conditions of the LaTeX Project Public License version 1.3c,
% available at http://www.latex-project.org/lppl/.

\documentclass[11pt,letterpaper,serif]{moderncv}

\usepackage{verbatim}

% moderncv themes
\moderncvstyle{classic}
\moderncvcolor{blue}
\frenchspacing
\usepackage{lastpage}

% character encoding
\usepackage[utf8]{inputenc}

% adjust the page margins
\usepackage[scale=0.8]{geometry}

% --- Begin use with Multibib ----
%\usepackage[resetlabels]{multibib}
%\usepackage[sorting=none]{multibib}
%\newcites{book,article,poster}{{Books},{Journal Articles},{Conference Posters}}
% Number the Publications
\makeatletter
\renewcommand*{\bibliographyitemlabel}{\@biblabel{\arabic{enumiv}}}
\makeatother
\renewcommand*{\bibliographyitemlabel}{[\arabic{enumiv}]}
% --- END use with Multibib ---

% BibTeX - don't print the year for publications that are "in press"
\newcommand{\dontprintyear}[1]{}

% personal data
\firstname{Lincoln J.}
\familyname{Harris}
\title{Curriculum Vitae}                        
\address{William H. Foege Hall}{3720 15th Ave NE}{Seattle, WA, 98195}
\email{lincolnh@uw.edu}                 
\social[github]{lincoln-harris}                             
%\quote{Last Updated: \today}    

\makeatletter
%\renewcommand*{\bibliographyitemlabel}{\@biblabel{\arabic{enumiv}}}
\makeatother

\begin{document}

\maketitle
\rfoot{\addressfont\itshape\textcolor{gray}{Page \thepage\ of \pageref{LastPage}}}

\cvline{}{Computational biologist with interests in machine learning, computational proteomics, scientific software development, outreach and open access science.}

\section{Education}
\cventry{2020-Present}{PhD, Genome Sciences}{University of Washington}{Seattle, WA}{}{Degree in progress.}
\cventry{2017}{B.A., Biology}{Swarthmore College}{Swarthmore, PA}{}{}

\section{Experience}

\cventry{2021-Present}{Bill Noble Laboratory}{University of Washington Department of Genome Sciences}{Seattle, WA}{}{
I utilize machine learning techniques to explore problems in computational proteomics. I frequently use cluster computing, \texttt{pytorch} , \texttt{scikit-learn}, and GPU-accelerated computing.   \\\\
Projects:
\begin{itemize}
    \item Imputation of quantitative mass spectrometry data with Non-negative Matrix Factorization and deep neural networks. 
\end{itemize}
}

\bigbreak

\cventry{2017--2020}{Chan Zuckerberg Biohub}{Spyros Darmanis Group}{San Francisco, CA}{Research Associate II}{
Investigated large biological datasets with a variety of tools including Docker, Python (\texttt{pandas, SciPy, Matplotlib, scikit-learn}), AWS, and workflow management systems such as NextFlow. Worked as part of a collaborative team in a fast-paced, cutting-edge scientific environment.  \\\\
Was promoted from Research Associate I to Research Associate II.   \\\\
Projects:
\begin{itemize}
\item Built a software tool for fast and memory-efficient summarizing of vcf (variant calling format) file entries following a sequencing experiment (\url{https://github.com/czbiohub/cerebra}.)
\item Identified variants in single-cell RNA-seq patient samples as part of a larger effort to better characterize changes in the lung tumor microenvironment across disease progression. In collaboration with the Trevor Bivona lab of UCSF.
\item Investigated the gene expression and alternate splicing dynamics of neuroendocrine cells in the developing lung, in collaboration with the Christin Kuo lab of Stanford.
\end{itemize}
}

\bigbreak

\cventry{2016--2017}{Brad Davidson Laboratory}{Swarthmore College Biology Department}{Swarthmore, PA}{}{
Assembled the genome and embryonic transcriptome of \emph{Corella Inflata} (sea squirt). Helped construct an improved phylogenetic tree for the \emph{Tunicate} subphylum, in collaboration with the Joe Ryan Lab of the University of Florida Whitney Marine Station
}

% \section{Research Training}
% \cventry{2020--2021}{Research Rotations}{University of Washington Department of Genome Sciences}{Seattle, WA}{}{Rotated in the labs of \href{https://sites.google.com/site/harriskelley/home}{Kelley Harris}, \href{https://www.beliveau.io/}{Brian Beliveau} and \href{https://noble.gs.washington.edu/}{Bill Noble}. Worked on population genetic simulations of extinction events, automatic annotation of cellular features from high-resolution microscopy images, and methods development for computational proteomics, respectively.}{}

% \bigbreak

% \cventry{2015}{Nick Kaplinsky Laboratory}{Swarthmore College Biology Department}{Swarthmore, PA}{}{Conducted basic plant molecular genetics research and learned basic lab skills such as western blotting, RNA extraction and PCR}{}

\section{Honors and Awards}

\cvline{2023}{Cascadia Proteomics Symposium -- Best poster award}

\cvline{2022}{NIH Ruth L. Kirschstein Predoctoral Individual National Research Service Award (F31) -- Awardee}

\cvline{2021}{NSF Graduate Research Fellowship Program -- Honorable Mention}

\cvline{2016 -- 2019}{Sigma Xi National Science Honors Society}

%\section{Publications}

%\renewcommand{\refname}{Articles}
\nocite{*}
%\bibliographystyle{plain}
\bibliographystyle{unsrt}
\bibliography{publications}     % 'publications' is a BibTeX file
\bigbreak

% \renewcommand{\refname}{Posters}
% \nociteposter{*}
% \bibliographystyleposter{unsrt}
% \bibliographyposter{posters}    % 'posters' is a BibTeX file

% --- END for use with multibib package ----

\section{Talks}

\cventry{2023}{Evaluating proteomics imputation methods with improved criteria}{American Society for Mass Spectrometry Asilomar Conference on Computational Proteomics}{Monterey, CA}{}{Full length talk}{}
\bigbreak

\cventry{2023}{Evaluating proteomics imputation methods with improved criteria}{Cascadia Proteomics Symposium}{Seattle, WA}{}{Short Talk}{}
\bigbreak

\cventry{2019}{New methods for high-throughput variant call processing}{Beyond the Cell Atlas Conference}{Berkeley, CA}{}{Short Talk}{}
\bigbreak

\cventry{2019}{Single-cell characterization of cancer hallmarks in advanced stage Lung Adenocarcinoma}{Chan Zuckerberg Biohub Inter-lab Confab}{San Francisco, CA}{}{Short Talk}{}

\section{Software}

\cvline{\texttt{Lupine}}{Imputation of quantitative proteomics data using deep neural networks. Add some more shit here.}

\cvline{\texttt{Cerebra}}{Performs high-throughput summarizing of nucleotide and protein-level variants following a RNA sequencing experiment. 
\newline\url{https://github.com/czbiohub/cerebra} (58 stars on GitHub)}

\section{Teaching}

\cventry{2024}{Instructor of Record}{GENOME 559 Introduction to Statistical and Computational Genomics}{University of Washington}{ Seattle, WA}{This was a graduate course. I was the instructor!}{}

\cventry{2023}{Teaching Assistant}{GENOME 361 Fundamentals of Genetics and Genomics}{University of Washington}{Seattle, WA}{This was an undergraduate course.}{}

\section{Mentees}

\cventry{2023}{Youssf Hegazy}{Columbia University}{New York, NY}{}{Direct mentor to a summer Research Experience Undergraduate (REU) student. We did some cool things, like xxx, xxx and xxx.}{}

\cventry{2019}{Rohan Vanheusden}{Nueva High School}{San Mateo, CA}{}{Direct mentor to an advanced high school student with interests in computational biology for 11 weeks. We did some cool things, like xxx, xxx and xxx.}{}

\section{Leadership and Outreach}

\cventry{2023--Present}{Graduate Student Representative}{Teaching Committee}{University of Washington Department of Genome Sciences}{Seattle, WA}{Explain what I do in this role.}

\cventry{2023--Present}{Organizer}{Computational Proteomics Journal Club}{University of Washington Department of Genome Sciences}{Seattle, WA}{Explain what I do in this role.}

\cventry{2021--Present}{Instructor}{The Carpentries Organization}{}{}{Certified to lead Data and Software Carpentries workshops designed to provide scientists with coding and data analysis skills}{}

\cventry{2021--Present}{Volunteer}{Skype-A-Scientist}{}{}{Regularly give career day presentations for elementary school classrooms. I introduce my work as a computational biologist and answer questions about graduate school, science and medicine.}{}

\cventry{2022}{Volunteer}{ROOTS Community Shelter}{}{}{Describe some of the things I did at ROOTS.}{}

\cventry{2021--2022}{Graduate Student Representative}{Graduate Council}{University of Washington Department of Genome Sciences}{Seattle, WA}{Attended monthly faculty meetings as the co-representative for graduate student interests.}{}

\cventry{2021--2022}{Volunteer}{University of Washington Science Explorers}{Seattle, WA}{}{Organized, planned and led several basic science lessons for a local elementary school classroom.}{}

\cventry{2018--2020}{Instructor}{Cupcakes and Coding}{Chan Zuckerberg Biohub}{San Francisco, CA}{Volunteered as an instructor at weekly beginner-friendly coding sessions}{}

% CV Entry Template
%\cventry{Date}{Title}{Organization}{School}{City}{Description}{}


\end{document}